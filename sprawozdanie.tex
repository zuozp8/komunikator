\documentclass[10pt,a4paper]{article}
\usepackage{fullpage}
\usepackage{polski}
\usepackage[utf8x]{inputenc}
\usepackage{ucs}
\usepackage{amsmath}
\usepackage{amsfonts}
\usepackage{amssymb}

\newcommand{\kod}[1]{\textbf{\small{KOD: #1}}}

\author{Konrad Baumgart, Jan Borowski}
\title{Projekt SK}

\begin{document}

\maketitle

\section{Treść zadania}
	Projekt 4: Komunikator internetowy typu GG

\section{Protokół sieciowy}
	Każdy klient identyfikowany jest przez swój 2-bajtowy dodatni numer ID, używa swojego hasła do logowania.\\
	Użytkownik ma tylko 2 statusy: zalogowany i niezalogowany.
	
	\paragraph{Logi}
		Serwer wypisuje dane diagnostyczne na standardowe wyjście błędów.
	\paragraph{Zapewnienie odczytu pełnych paczek z danymi}
		Za każdym razem zarówno serwer jak i klient poprzedza przesyłane dane
		informacją o ich długości (w bajtach) zawartą w 2 bajtach. Nie wlicza się w tę długość
		2 bajtów przechowujących ilość danych.
		Ogranicza to na przykład maksymalną	długość wiadomości w systemie.\\
		Pierwszą częścią danych jest zawsze kod operacji o długości 1 bajta.\\
		Wszelskie ciągi znaków wysyłamy bez znaku 0 na końcu.\\
		Przesyłając liczby wysyłamy najpierw mniej znaczące bajty, potem bardziej znaczące.
	\paragraph{Postępowanie w przypadku otrzymania nieprawidłowych danych}
		Gdy serwer otrzyma od klienta dane niezgodne z niniejszą specyfikacją, rozłącza danego klienta.
	\paragraph{Port serwera} Serwer nasłuchuje pakietów TCP na porcie 4790.
	\paragraph{Łączenie} Nawiązując połączenie z serwerem klient może albo się zarejestrować, albo zalogować.
	\paragraph{Rejestracja}\kod{1}\\
		Można się rejestrować jedynie tuż po nawiązaniu połączenia.
		Klient przesyła swoje hasło, musi ono mieć conajmniej 3 bajty(znaki).
		Serwer odpowiada \kod{1} wysyłając 2 bajty: numer ID w przypadku następnie rozłącza się.
	\paragraph{Logowanie}\kod{2}\\
		Klient wysyła swoje ID (2 bajty) do serwera, po nim zaś swoje hasło.
		Serwer odpowiada \kod{2} wysyłając 1 bajt: 1 w przypadku sukcesu lub kończy połączenie w przypadku błędu.
	\paragraph{Zakończenie połączenia} Zarówno klient jak i serwer mogą zakończyć połączenie po prostu
		kończąc połączenie TCP.
	
	\paragraph{} \textbf{Poniższe komendy można wykonać tylko będąc zalogowanym.}
	\paragraph{Sprawdzenie dostępności znajomych}\kod{3}\\
		Klient wysyła kilka ID (2-bajtowe, jedno po drugim) do serwera.
		Serwer odpowiada \kod{3} wysyłając wielokrotność 3 bajtów: dla każdego ID o które wystosowano zapytanie
		2 bajty zajmuje to ID, zaś w trzecim bajcie jest kod dostępności danego użytkownika.\\
		Można sprawdzić dostępność samego siebie.\\
		Można sprawdzić dostępność maksymalnie 1000 osób.
	\paragraph{Wysłanie wiadomości}\kod{5}\\
		Klient wysyła 2 bajty - ID odbiorcy, a potem wysyłaną wiadomość\\
		Serwer nie informauje nadawce o powodzeniu operacji wysłania.\\
		Serwer wysyła \kod{5} do odbiorcy 2 bajty - ID nadawcy, a potem wysyłaną wiadomość.\\
		Jeżeli odbiorca jest niezalogowany, wiadomości do niego są przechowywane na serwerze
		i są mu przesyłane w momencie gdy się zaloguje.
		
\section{Realizacja}
	\subsection{Serwer}Stworzyliśmy serwer używając poznanych na zajęciach socketów BSD używając języka C++.
	By przechowywać dane o użytkownikach, używamy dynamicznych kontenerów z STL.
	Kod serwera jest krótki, więc pozwoliliśmy sobie go zamknąć w pojedynczym pliku \textit{main.cpp}.
	\begin{quote}
		\textit{Code is the ultimate documentation.}
	\end{quote}
	\paragraph{Wielowątkowość}Mieliśmy problem z wieloątkowością - gdybyśmy dla każdego połączenia tworzyli
	nowy wątek, to z tegoż nowego wątku nie moglibyśmy bezpośrednio wysyłać wiadomości do użytkowników, którzy
	się zalogowali po nas - nie mielibyśmy dostępu do gniazd dla nich stworzonych. Dlatego nie tworzyliśmy nowych
	wątków, a użyliśmy funckji \textit{select} by oczekiwać na możliwość odczytu lub zapisu do wielu gniazd.
	\paragraph{Buforowanie wiadomości}Aby można było wysyłać wiadomości do niezalogowanych użytkowników, na serwerze
	trzymamy je na mapie \textit{bufferedMessages} w postaci kolejki całych pakietów do wysłania identyfikowanej przez
	ID użytkownika, do którego mają zostać wysłane. Jeżeli odbiorca zerwie połączenie gdy przesyłamy mu wiadomość, to
	zostaje ona utracona.
	\paragraph{}Odczytując dane z gniazda lub zapisując do niego, możemy nie odczytać/zapisać całego pakietu.
	Dlatego też zawsze gdy odczytujemy dane z gniazdka, wpisujemy je do \textit{readBuffor}a, później zaś, gdy
	odbierzemy cały pakiet, kopiujemy go i usówamy z readBuffora. Podobnie, wysyłając dane do klienta, zapisujemy je do
	\textit{writeBuffor}a i staramy się sukcesywnie przesyłać.
	
	\subsection{Klient dla systemu Windows}Klienta stworzyliśmy w języku Java, gdyż byliśmy w stanie potem uruchamić go także w środowisku
	Linux. Do stworzenia GUI użyliśmy biblioteki SWING. Program składa się z dwóch głównych okienek: jednego 
	zawierającego listę kontaktów i drugiego zawierającego rozmowy.
	\paragraph{Implementacja}
	\begin{quote}
		\textit{GlowneOkno.java} Klasa rozpoczynająca program i zarządzająca jego głównymi elementami. \\
		\textit{Kontakt.java} Klasa przechowująca informacje o Kontakcie tzn. jego nick, id, aktualną dostępność i czy 
			toczy się z nim rozmowa.\\
		\textit{ListaKontakow.java} Przewowuje i wyświetla listę kontaktów. \\
		\textit{OknoDodania.java} Przyjmuje i zwraca dane nowego kontaktu. \\
		\textit{OknoEdycji.java} Umożliwia zmianę nicku znajomego. \\
		\textit{OknoLogowania.java} Przyjmuje dane do logowania/rejestracji. \\
		\textit{OknoRozmowy.java} Klasa zarządzająca oknem z rozmowami. \\
		\textit{PanelRozmowy.java} Zarządza pojedyńczą rozmową, zawiera okno wpisywania wiadomości, okno rozmowy i przycisk wysłania. \\
		\textit{WatekSieciowy.java} Zarządza połączeniem z serwerem, nadawaniem i odbieraniem wiadomości wszystkich typów.
			Składa się z dwóch wątków. Jeden z nich wysyła dane, a drugi wczytuje dane.\\
		\textit{Wiadomosc.java} Reprezentuje pojedyńczą wiadomość przesyłaną w rozmowach. \\
		\textit{ZnajomiCellRenderer.java} Klasa pomocnica służąca do wyświetlania nicku oraz ikony dostępności w liście kontaków. 
	\end{quote} 
	\paragraph{Główne problemy} Największym problemem napotkanym w czasie implementacji było poprawne odczytywanie stanu połączenia
		tzn. czy jeszcze jesteśmy połączeni czy serwer się rozłączył. Początkowo istniał tylko jeden wątek do komunikacji sieciowej, który
		najpierw odbierał dane, a potem je wysyłał. Korzystaliśmy wtedy z klasy \textbf{Socket} z pakietu \textit{java.net.Socket},
		niestety to rozwiązanie okazało się nie wystarczające. Nie mogliśmy poprawnie rozwiązać problemu błędnego logowania
		(zgodnie ze specyfikacją serwer rozłącza się gdy podamy złe hasło) i 
		rozłączenia serwera w czasie pracy klienta. \\ \\
		Okazało się, że lepszym wyjściem było użycie klasy \textbf{SocketChannel} z pakietu \textit{java.nio.channels.SocketChannel}
		oraz klasy \textbf{Selector} z pakietu \textit{java.nio.channels.Selector}, 
		ale w pełni satysfakcjonujące rozwiązanie otrzymaliśmy gdy wyodrębniliśmy wysyłanie komunikatów jako osobny wątek,
		musieliśmy wtedy też zacząć używać semaforów, by uniknąć współbieżnej modyfikacji tych samych danych.\\
		Dzięki usprawnieniom program działa nawet przy bardzo niewielkich buforach gniazd tcp.
	\subsection{Klient dla systemu Android}
		Nie mieliśmy żadnego doświadczenia w tworzeniu aplikacji na urządzenia mobilne.\\
		Klient dla systemu Android został nazwany \textbf{Komunikator ROTFL}, kod umieściliśmy w pakiecie \textit{your.mojemojelol}.\\
		Współdzieli z desktopowym klientem cześć kodu związaną z komunikacją sieciową.\\
		GUI tworzyliśmy używając API systemu android, więc kod związany z graficznym interfejsem użytkownika musieliśmy  przepisać.
	
\section{Obsługa programu}
	\subsection{Serwer}
		Skompilowany przy (użyciu programu make) serwer uruchamia się w konsoli.
		Po uruchomieniu można obserwować wyjście diagnostyczne na standardowym wyjściu błędów.
		By łagodnie wyłączyc serwer wystarczy wysłać doń SIGUSR1.
	\subsection{Klient}
		Klient uruchamiany jest na maszynie wirtualnej Java.
		Intuicyjny graficzny interfejs użytkownika pozwala zarejestrować się, zalogować i kożystać z komunikatora.

\end{document}
